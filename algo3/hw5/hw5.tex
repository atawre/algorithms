\documentclass{letter}
\usepackage{enumerate}

%%%%%%%%%% Start TeXmacs macros
\newcommand{\section}[1]{\medskip\bigskip

\noindent\textbf{\LARGE #1}}
\newcommand{\tmstrong}[1]{\textbf{#1}}
\newcommand{\tmtextit}[1]{{\itshape{#1}}}
\newenvironment{enumeratenumeric}{\begin{enumerate}[1.] }{\end{enumerate}}
\newenvironment{itemizedot}{\begin{itemize} \renewcommand{\labelitemi}{$\bullet$}\renewcommand{\labelitemii}{$\bullet$}\renewcommand{\labelitemiii}{$\bullet$}\renewcommand{\labelitemiv}{$\bullet$}}{\end{itemize}}
\newenvironment{itemizeminus}{\begin{itemize} \renewcommand{\labelitemi}{$-$}\renewcommand{\labelitemii}{$-$}\renewcommand{\labelitemiii}{$-$}\renewcommand{\labelitemiv}{$-$}}{\end{itemize}}
%%%%%%%%%% End TeXmacs macros

\begin{document}

\title{CS157 Homework 5}\author{chaoqian and silao\_xu}\maketitle

\section{Problem 1}

Bilbo has a long journey ahead of him and wants to stay in comfortable hotels
along his route each night, but can only walk 20 miles in a day. Fortunately,
he has a guidebook charting the locations of all the hotels along his route.
(Bilbo already knows which route he is taking; he only has to choose where
along his route to stay each night.)
\begin{enumeratenumeric}
  \item (8 points) Find a greedy algorithm for Bilbo to compute how to finish
  his journey in the least number of days. Points on this problem will only be
  given for the proof that your algorithm is optimal; more points will be
  given for simpler and clearer proofs.
  \begin{itemizedot}
    \item \tmtextit{{\tmstrong{Greedy Algorithm}}}
    
    Let start point be $s$ and the termination be $T$. The following algorithm
    will find the next hotel as far as possible within 20 miles for each hop.
    \begin{verbatim}
 1   def ():
 2       ()
 3       if  then do:
 4          return 0
 5       else:
 6          return ()
 7   end
 8
 9   def ():
10       .
11       while .. then do:
12           
13           .
14       end
15       return 
16   end
\end{verbatim}
    \item {\tmstrong{\tmtextit{Correctness}}}
    \begin{itemizeminus}
      \item {\tmstrong{Claim}}. \tmtextit{FindWay} terminates.
      
      \tmtextit{Proof}: The \tmtextit{FindWay} would recurse on the next route
      point (the hotel along the route he will live in at night). Since the
      destination of the route is known, therefore line 3 would become true at
      some point.
      
      \item {\tmstrong{Claim}}. \tmtextit{FindWay} satisfies the feasibility
      that he needs to walk at most 20 miles every day in the trip.
      
      \tmtextit{Proof}:
      
      \item {\tmstrong{Claim}}. \tmtextit{FindWay} satisfies the optimality
      criteria that he can finish his journey in the least number of days.
    \end{itemizeminus}
    
    
    
  \end{itemizedot}
  
  
  
  
  \item (3 points) After studying your algorithm for the previous problem,
  Bilbo realizes that, actually, the different hotels cost different amounts,
  and what he actually wants to do is minimize the total cost of his journey.
  (Luckily, his guidebook also lists the cost of each hotel.) He thinks of the
  following greedy algorithm: wherever he is, for each hotel within a day's
  walk of him (20 miles), he computes the ``cost per mile'' of staying there,
  dividing its cost by the amount of progress he would make by staying there;
  given this list of costs, he then chooses to spend his next night at the
  hotel with the best cost per mile. Demonstrate for Bilbo that being greedy
  can be costly, that is, describe an example where Bilbo's algorithm gives
  suboptimal performance.
  
  \\
  {\linebreak}{\linebreak}{\linebreak}{\linebreak}\\
  {\linebreak}{\linebreak}{\linebreak}{\linebreak}\\
  {\linebreak}{\linebreak}{\linebreak}{\linebreak}\\
  {\linebreak}{\linebreak}{\linebreak}{\linebreak}\\
  {\linebreak}{\linebreak}{\linebreak}{\linebreak}\\
  {\linebreak}{\linebreak}{\linebreak}{\linebreak}\\
  {\linebreak}{\linebreak}{\linebreak}{\linebreak}\\
  \\
  {\linebreak}{\linebreak}{\linebreak}{\linebreak}{\linebreak}{\linebreak}{\linebreak}{\linebreak}\\
  {\linebreak}{\linebreak}{\linebreak}{\linebreak}\\
  {\linebreak}{\linebreak}{\linebreak}{\linebreak}\\
  {\linebreak}{\linebreak}{\linebreak}{\linebreak}\\
  {\linebreak}{\linebreak}{\linebreak}{\linebreak}\\
  {\linebreak}{\linebreak}{\linebreak}{\linebreak}\\
  {\linebreak}{\linebreak}{\linebreak}{\linebreak}\\
  {\linebreak}{\linebreak}{\linebreak}{\linebreak}\\
  {\linebreak}{\linebreak}{\linebreak}{\linebreak}\\
  {\linebreak}{\linebreak}{\linebreak}{\linebreak}\\
  {\linebreak}{\linebreak}{\linebreak}{\linebreak}\\
  {\linebreak}{\linebreak}{\linebreak}{\linebreak}\\
  {\linebreak}{\linebreak}{\linebreak}{\linebreak}\\
  {\linebreak}{\linebreak}{\linebreak}{\linebreak}
  
  \item (4 points) Find a dynamic programming algorithm for Bilbou2019s
  problem. Make it clear to Bilbo why it works, including an explanation of
  the meaning of any tables you ask Bilbo to construct. (Your solution for
  this part should look like an explanation, not a proof.)
  
  \\
  {\linebreak}{\linebreak}{\linebreak}{\linebreak}\\
  {\linebreak}{\linebreak}{\linebreak}{\linebreak}\\
  {\linebreak}{\linebreak}{\linebreak}{\linebreak}\\
  {\linebreak}{\linebreak}{\linebreak}{\linebreak}\\
  {\linebreak}{\linebreak}{\linebreak}{\linebreak}\\
  {\linebreak}{\linebreak}{\linebreak}{\linebreak}\\
  {\linebreak}{\linebreak}{\linebreak}{\linebreak}\\
  {\linebreak}{\linebreak}{\linebreak}{\linebreak}\\
  {\linebreak}{\linebreak}{\linebreak}{\linebreak}\\
  {\linebreak}{\linebreak}{\linebreak}{\linebreak}\\
  {\linebreak}{\linebreak}{\linebreak}{\linebreak}\\
  {\linebreak}{\linebreak}{\linebreak}{\linebreak}\\
  {\linebreak}{\linebreak}{\linebreak}{\linebreak}\\
  {\linebreak}{\linebreak}{\linebreak}{\linebreak}\\
  {\linebreak}{\linebreak}{\linebreak}{\linebreak}\\
  {\linebreak}{\linebreak}{\linebreak}{\linebreak}\\
  {\linebreak}{\linebreak}{\linebreak}{\linebreak}\\
  {\linebreak}{\linebreak}{\linebreak}{\linebreak}\\
  {\linebreak}{\linebreak}{\linebreak}{\linebreak}\\
  {\linebreak}{\linebreak}{\linebreak}{\linebreak}\\
  {\linebreak}{\linebreak}{\linebreak}{\linebreak}\\
  {\linebreak}{\linebreak}{\linebreak}{\linebreak}\\
  {\linebreak}{\linebreak}{\linebreak}{\linebreak}\\
  {\linebreak}{\linebreak}{\linebreak}{\linebreak}\\
  {\linebreak}{\linebreak}{\linebreak}{\linebreak}\\
  {\linebreak}{\linebreak}{\linebreak}{\linebreak}\\
  {\linebreak}{\linebreak}{\linebreak}{\linebreak}\\
  {\linebreak}{\linebreak}{\linebreak}{\linebreak}
\end{enumeratenumeric}

\end{document}
