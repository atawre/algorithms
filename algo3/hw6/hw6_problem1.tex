\documentclass{letter}

%%%%%%%%%% Start TeXmacs macros
\newcommand{\emdash}{---}
\newcommand{\section}[1]{\medskip\bigskip

\noindent\textbf{\LARGE #1}}
\newcommand{\tmem}[1]{{\em #1\/}}
\newcommand{\tmstrong}[1]{\textbf{#1}}
\newenvironment{itemizedot}{\begin{itemize} \renewcommand{\labelitemi}{$\bullet$}\renewcommand{\labelitemii}{$\bullet$}\renewcommand{\labelitemiii}{$\bullet$}\renewcommand{\labelitemiv}{$\bullet$}}{\end{itemize}}
%%%%%%%%%% End TeXmacs macros

\begin{document}

\title{CS157 Homework 6}\author{tqian and silao\_xu\\
}\maketitle

\section{Problem 1}

(30 points) You can allocate a block of n memory locations on your computer in
constant time, however the contents of the memory in the block may be
arbitrary. Typically, you will initialize these memory locations before you
use them, by setting them all to a special symbol Empty, which takes $O$($n$)
time.

The object of this problem is to create a new data structure that mimics the
properties of an array, while being much faster to initialize, but while still
ensuring that any values returned by this data structure are meaningful, and
not uninitialized garbage.

You need to come up with a data structure that behaves like a 0-indexed array
$A$ of n elements. The following operations must take a constant time:
\begin{itemizedot}
  \item set(index, value): Assign the value to $A$[index].
  
  \item get(index): Return the value from $A$[index]. If no value has yet been
  assigned, return Empty.
  
  \item initialize(n): Initialize the data structure so that it will mimic an
  array of size $n$, and where any {\tmstrong{get}} call returns
  {\tmstrong{Empty}}.
\end{itemizedot}
{\tmstrong{Warning}}: Keep in mind that, initially, the entries in memory can
be arbitrary and may imitate valid parts of whatever data structure you
design{\emdash}your data structure should work {\tmem{no matter what}} is in
memory initially.

{\tmstrong{Hints}}:
\begin{itemizedot}
  \item Use more than $n$ storage (but you do not need to use more than
  $O$($n$) storage).
  
  \item Most memory locations may be garbage, but think about how you can be
  sure some memory locations are meaningful.
  
  \item Because all operations in this data structure must take constant
  (worst case) time, you cannot use anything fancy: no hash tables, no binary
  search trees, no heaps, etc.
\end{itemizedot}
{\tmstrong{Important}}: If you are using extra space, please explain in
sentences how they are used. As always, you need to communicate clearly that
your proposed data structure works correctly, and that the running time for
each operation, including initialization, is constant.{\tmem{{\tmstrong{}}}}


\begin{itemizedot}
  \item initialize(n): Initialize the data structure so that it will mimic an
  array of size $n$, and where any {\tmstrong{get}} call returns
  {\tmstrong{Empty}}.
  
  \begin{verbatim}
():
    new array of length 
    new array of length  /* for recording assignment sequence */
    new array of length  /* for tracing back the correct index */
     /* for counting how many entries have been initialized */
\end{verbatim}
  
  \item get(index): Return the value from $A$[index]. If no value has yet been
  assigned, return Empty.
  
  \begin{verbatim}
():
    if  and  then do:
        return 
    else:
        return 
\end{verbatim}
  
  \item Pseudocode for set operation is as follows.
  
  \begin{verbatim}
(, ):
    if () then do:
        
    else:
        
        
        
        
\end{verbatim}
\end{itemizedot}


\end{document}
